
% PLEASE USE UTF-8 ENCODING WHEN EDITING THIS FILE!

% WWW 2012 page limit: 10 pages

%\documentclass{llncs}
% \documentclass{acm_proc_article-sp}
\documentclass{sig-alternate}  % tighter style --> less pages (use the other one if we have enough space)
%\documentclass{www2012-accepted} 
%\usepackage{amsmath, amssymb}
%\usepackage{graphicx}
\usepackage{hyperref}
%\usepackage{listings}
\usepackage[utf8]{inputenc}
\usepackage{moreverb}

\newcommand{\todo}[1]{\textbf{ToDo: \textit{#1}}}
%\newcommand{\comm}[1]{\textbf{Comment: \textit{#1}}}\stackrel{\leftrightarrow}{}

\usepackage{enumitem}
\usepackage[usenames,dvipsnames]{color}
\newcommand{\sparqlwo}[2]{{\texttt{#1 \char '173} \\[1.5pt] \hspace*{.5cm}\parbox{6cm}{\tt #2} \\ \texttt{\char '175} }}
\newcommand{\sparql}[3]{{\texttt{#1 \char '173} \\[1.5pt] \hspace*{.5cm}\parbox{6cm}{\tt #2} \\ \texttt{\char '175} \\ \texttt{#3} }}
\newcommand{\slot}[3]{$\langle\texttt{#1},\text{#2},\text{\sf #3}\rangle$}
\newcommand{\argmax}{\operatornamewithlimits{arg\,max}}
\newcommand{\argmin}{\operatornamewithlimits{arg\,min}}

\usepackage{qtree}
\usepackage[usenames,dvipsnames,table]{xcolor}


\newcommand{\Drs}[2]{%%%%%%%%%%%%%%%%%%%%%%%%%
\(                   % begin maths mode
 \begin{array}{|l|}  %
 \hline              % top line
   \begin{array}{l}  %
    #1                % `Universe'
    \end{array} \\   %  end the `universe' part
 \hline              % line between Universe and Conditions
    \begin{array}{l} %
    #2                % the conditions
    \end{array} \\   % end the conditions part
 \hline              % bottom line
\end{array}          %
\)                   % end maths mode
                     }%%%%%%%%%%%%%%%%%%%%%%%%%


\begin{document}

\clubpenalty=10000 
\widowpenalty = 10000

%\conferenceinfo{WWW}{2012 Lyon, France}

\title{Duplicate-Aware Federated Query Processing over the Data Web}

\numberofauthors{2} 
\author{
% 1st. author
\alignauthor
Muhammad Saleem\\
       \affaddr{DERI Galway}\\
       \affaddr{??}\\ 
       \email{??}
% 2nd. author
\alignauthor
Axel-Cyrille Ngonga Ngomo\\
       \affaddr{Universit\"at Leipzig, IFI/AKSW}\\
       \affaddr{PO 100920, D-04009 Leipzig}\\
       \email{ngonga@informatik.uni-leipzig.de}

}

%\date{30 July 1999}
\maketitle

%\title{SPARQL Template Based Question Answering}
% \titlerunning{SPARQL Template Based Question Answering}  
%\author{Christina Unger\inst{1} \and Lorenz B{\"u}hmann\inst{2} \and Jens Lehmann\inst{2} \and Philipp Cimiano\inst{1}}
%\authorrunning{...} 
%\institute{
%Bielefeld University, CITEC, Semantic Computing Group\\
%Universit\"atsstra{\ss}e 21--23, 33615 Bielefeld\\ 
%\email{cunger|cimiano@cit-ec.uni-bielefeld.de}
%\and 
%University of Leipzig, Computer Science Institute, AKSW Group,
%\\Johannisgasse 26, 04103 Leipzig\\
%\email{buehmann|lehmann@informatik.uni-leipzig.de}
%}
\begin{abstract}

\end{abstract}

% A category with the (minimum) three required fields
\category{H.2.4}{Database Management}{Systems}[Distributed databases]
%A category including the fourth, optional field follows...
% \category{D.2.8}{Software Engineering}{Metrics}[complexity measures, performance measures]

\terms{Algorithms, Experimentation, Theory}

\keywords{Federated queries, SPARQL, deduplication}

\section{Introduction (AN+MS)}
\section{Related Work (MS)}
\subsection{Federated SPARQL queries}
\subsection{Filters}
Bloom
MIPS
etc.
\section{Notation}
In this section, we present the core of the notation that will be used throughout this paper.
We denote data sources with $S$ and the total number of data sources with $n$. 
The set of all possible result sets is denoted $R$ while the set of all possible SPARQL queries is labeled with $Q$. 
A data source ranking function $rank: S \times Q \rightarrow \{1 \ldots n\}$ is a function that assigns a ranking to each data source given a particular query $q \in Q$.
Note that for any source ranking function $rank$, we assume that $\forall S, S' \forall q \in Q : S  \neq S' \Longleftrightarrow rank(S, q) \neq rank (S', q)$.
A result set estimation function $est: S \times Q \rightarrow \mathbb{N}$ aims at approximating the size of the result set that will be returned by a given query.
Note that this function plays a crucial role in the processing of federated queries as it is most commonly used to decide upon the ranking of data sources for a given query.
The aim of a federated query system such as the one described in this work is thus to optimize its estimation function $est$ so as to ensure a ranking of the source close to the optimal ranking. %do we really need this sentence?
\section{Approach (MS)}
\subsection{Overview}
\subsection{MIPS}
Vector Construction
Union, overlap
\subsection{Index construction}
Compression Ratio Setting
\subsection{Source Selection}
\subsection{Source Ranking}
Result set estimation       
\subsection{Subquery generation}
\section{Experiments and Results (AN)}
\subsection{Experimental Setup (AN)}
\subsubsection{Datasets (3 datasets)}
\subsubsection{Queries (6 types)}
\subsubsection{Metrics (MSE)}
\subsection{Results (AN)}
\subsubsection{Ranking Error}
\subsubsection{Result Set Estimation Error}
\subsubsection{Execution Time}
\subsubsection{Size of MIPS vectors (on largest dataset)}
\section{Discussion (AN)}



\bibliographystyle{plain}
\bibliography{literature}


\end{document}
